\documentclass{beamer}
\usetheme{metropolis}

\usepackage{xcolor}
\usepackage{graphicx}
\usepackage{tcolorbox}
\usepackage{listings}
\usepackage{booktabs}
\usepackage{array}
\usepackage{tabularx}
\usepackage{lmodern}
\usefonttheme[onlymath]{serif}

\begin{document}
\newcommand{\tabitem}{~~\llap{\textbullet}~~}

\definecolor{codegreen}{rgb}{0,0.6,0}
\definecolor{codegray}{rgb}{0.5,0.5,0.5}
\definecolor{codepurple}{rgb}{0.58,0,0.82}
\definecolor{backcolour}{rgb}{0.95,0.95,0.92}

\lstset{
	language=C++,
	basicstyle=\scriptsize\ttfamily,
	keywordstyle=\color{magenta},
	stringstyle=\color{codepurple},
	numbers=left,
	numbersep=5pt,
	numberstyle=\tiny\color{codegray},
	backgroundcolor=\color{backcolour},
	showstringspaces=false,
	tabsize=4
}


\title{Brute Force: Multiple Queries}
\subtitle{It makes all the difference}
\author{UTEC - Competitive Programming}
\date{}

\maketitle

\begin{frame}
	\frametitle{Challenge}

	Given a number $n$, determine if it is prime.

	\begin{itemize}
		\item<1-> Design an algorithm for  $n \leq 10^8$
		\item<2-> Now make one for $n \leq 10^{16}$
		\item<3-> What if $n \leq 10^7$, but there are $q$ ($q \leq 10^7$) queries.
	\end{itemize}
\end{frame}

\begin{frame}[fragile]
	\frametitle{Sieve of Eratosthenes}

	\begin{itemize}
		\item The following is a very famous sieve for calculating what numbers are prime.
		\item This sieve has a complexity of $O(nlg(lgn))$. \textit{Why?}
	\end{itemize}

	\lstinputlisting{listings/sieve.cpp}
\end{frame}

\begin{frame}
	\frametitle{Preprocessing}

	\begin{itemize}
		\item When we have multiple queries, we can sometimes compute some values that will help us answer queries.
		\item This is called preprocessing and the idea is that it reduces the time per query without being 
		\item Lets say $T_p(n)$ is the complexity of preprocessing and $T_q(n)$ is the complexity per query after said preprocessing, the complexity of our program would be $O(max(T_{p}(n), Q T_{q}(n)))$.
	\end{itemize}
\end{frame}

\begin{frame}
	\frametitle{Challenge - Prefix Sums}

	You are given an array $a$ of size $n$. You will receive $q$ queries that will consist of to numbers $l$ ans $r$. For each you need to print $\sum\limits_{i = l}^{r}a_i$.

	\begin{itemize}
		\item<1-> Design an algorithm to solve this for $n,q \leq 10^4$.
		\item<2-> Can we solve this problem for $n,q \leq 10^6$?
	\end{itemize}
\end{frame}

\begin{frame}[fragile]
	\frametitle{Solution - Prefix Sums}

	\begin{itemize}
		\item By preprocessing the array and calculating the sum for any prefix we can answer queries with a complexity of O(1). 
		\item The final complexity for our solution would be O(min(n, q)).
	\end{itemize}

	\lstinputlisting[basicstyle=\tiny\ttfamily]{listings/prefix.cpp}
\end{frame}

\begin{frame}
	\frametitle{Challenge - Offline Processing}

	You are given an array of size $m$ and $q$ queries. Each consists of 2 numbers $n$ ans $p$. For each query output $|\{x : x \leq n \wedge p|a[x]\}|$. $q \leq 10^6$ and $p \leq n \leq 10^6$

	\begin{itemize}
		\item Any ideas?
	\end{itemize}
\end{frame}

\begin{frame}
	\frametitle{Offline Processing}

	\begin{itemize}
		\item In many cases the order in which we answer the queries doesn't affect their answers.
		\item Offline processing is changing the order of queries in order to improve overall speed.
		\item \textbf{IMPORTANT:} In order to be able to process queries offline, all queries must be independent (The result of a query is not affected by previous queries).
	\end{itemize}
\end{frame}

\begin{frame}
	\frametitle{Solution - Offline Processing}

	\begin{itemize}
		\item We can use a sieve to find all the divisors of a number.
		\item Order the queries in ascending order according to their $n$.
		\item Initialize counter \texttt{cnt}. \texttt{cnt[i]} stores how many values are divisible by $i$.
		\item Iterate $x$ through all numbers from 1 to $n_{max}$ and increase the counter of all divisors of $x$ by one.
		\item Answer all queries where $n_i = x$.
	\end{itemize}
\end{frame}

\section{Thanks for Listening!}

\end{document}
