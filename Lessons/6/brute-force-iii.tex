\documentclass{beamer}
\usetheme{metropolis}

\usepackage{xcolor}
\usepackage{graphicx}
\usepackage{tcolorbox}
\usepackage{listings}
\usepackage{booktabs}
\usepackage{array}
\usepackage{tabularx}
\usepackage{lmodern}
\usefonttheme[onlymath]{serif}
\usepackage{amsmath}

\begin{document}
\newcommand{\tabitem}{~~\llap{\textbullet}~~}

\definecolor{codegreen}{rgb}{0,0.6,0}
\definecolor{codegray}{rgb}{0.5,0.5,0.5}
\definecolor{codepurple}{rgb}{0.58,0,0.82}
\definecolor{backcolour}{rgb}{0.95,0.95,0.92}

\lstset{
	language=C++,
	basicstyle=\scriptsize\ttfamily,
	keywordstyle=\color{magenta},
	stringstyle=\color{codepurple},
	numbers=left,
	numbersep=5pt,
	numberstyle=\tiny\color{codegray},
	backgroundcolor=\color{backcolour},
	showstringspaces=false,
	tabsize=4
}


\title{Brute Force: Common Strategies}
\subtitle{Keep it simple, keep it neat}
\author{UTEC - Competitive Programming}
\date{}

\maketitle

\section{Fixing Variables}

\begin{frame}
	\frametitle{Fixing Variables}

	\begin{itemize}
		\item One of the most intuitive techniques.
		\item Usually this problems involve an equation of $m$ variables.
		\item You will \textit{fix} a variable by making it constant and then try to find the solution under this new constrain.
		\item You will iterate through all possible values of the fixed variable and see how this fixed solution contributes to the general solution of the problem.
	\end{itemize}
\end{frame}

\begin{frame}[fragile]
	\frametitle{Fixing Variables - Example}

	Given numbers $n$, $a$, $b$ and $c$ find:
	$$|\{(x, y, z) | x, y, z \in \mathbb{Z}^{+0}
	\wedge x \leq a \wedge y \leq b \wedge z \leq c
	\wedge \frac{x}{2} + y + 2z = n\}|$$

	\begin{itemize}
		\item<2-> Fix $x$ and treat it as a constant: $y + 2z = n_0$.
		\item<3-> Fix $y$: $2z = n_1$
		\item<4-> Check if there exists a value of $z$ that satisfies the equation.
		\item<5-> This soultion has a complexity of $O(ab)$. Can we do better?
	\end{itemize}
\end{frame}

\section{Simulation with Brute Force}

\begin{frame}
	\frametitle{Simulation}

	\begin{itemize}
		\item Problems that can be solved using this strategy usually have an underlying pattern or rules that we can compute easily.
		\item To find the answer we just need to let our simulation run until we get the expected results.
		\item \textbf{Problem:} Consider the following function:
			$$ f(n) = \begin{cases} \frac{n}{2} & n \equiv 0 \pmod{2}\\
			3n + 1 & \text{otherwise} \end{cases}$$

			The cycle of a number $n$ is the number of times we need to apply function $f$ so that $n = 1$. Find the maximum cycle between two given numbers $i$ and $j$. $i, j < 10^5$
	\end{itemize}
\end{frame}

\begin{frame}
	\frametitle{Simulation: Example}

	\begin{itemize}
		\item \textit{Collatz Conjecture:} It's guaranteed that $n$ will reach $1$.
		\item<2-> Make a function \texttt{count(x)} that counts how many times we need to apply a $f$ to $x$ get $1$:
		\item<3-> Run that function for all $x$ between $i$ and $j$ and save the maximum.
		\item<4-> Complexity? Smallest cycle smaller than $10^{5} \approx 400$, therefore $O(400m)$. Where $m = max\{i - j\}$
	\end{itemize}
\end{frame}

\section{Weak Constrains}

\begin{frame}
	\frametitle{Weak Constrains}

	\begin{itemize}
		\item This problems usually seem hard until you realize an underlying constrain that reduces significantly the search space.
		\item Even the problem might have some constrains, this might be very far from the constrains that can be deduced by the statement.
		\item After finding this new constrains we can use simple brute-force.
	\end{itemize}
\end{frame}

\begin{frame}
	\frametitle{Weak Constrains: Example}

	Given $n$ and $s$ ($n,s \leq 10^{18}$), find:
	$$|\{x | x \leq n \wedge x - f(x) \geq s\}|$$
	Where $f(x)$ is the sum of digits of $x$.

	\begin{itemize}
		\item Iterating through all possible values of $n$ will never work.
		\item<2-> Observation 1: $x < s \rightarrow x - f(x) < s$
		\item<3-> Observation 2: $f(x) \leq 9 \cdot 18$
		\item<4-> Observation 3: $x \geq s + 9 \cdot 18 \rightarrow x - f(x) \geq s$
		\item<5-> We only need to check $x \in [s, s + 9 \cdot 18]$!
		\item<5-> Answer will be \texttt{count}$(s, s + 9 \cdot 18) + n - (s + 9 \cdot 18)$.
	\end{itemize}
\end{frame}

\section{Thanks for Listening!}

\end{document}
